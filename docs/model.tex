\documentclass[a4paper,12pt]{article}
\usepackage[left=2.0cm,right=2cm,top=2cm,bottom=2cm]{geometry}
\usepackage{amsmath}
\usepackage{amsfonts}
\usepackage{graphicx}
\newenvironment{packed_enum}{
\begin{itemize}
  \setlength{\topsep}{0pt}
  \setlength{\itemsep}{0pt}
  \setlength{\parskip}{0pt}
  \setlength{\parsep}{0pt}
  \setlength{\partopsep}{0pt}
}{\end{itemize}}
\begin{document}
\section*{Simple model for $\textrm{R}_{0}$}
\begin{eqnarray*}
\textrm{j}(\textrm{t})&=&\int_{0}^{\infty}\textrm{A}(\tau)\textrm{j}(\textrm{t}-\tau)d\tau \\
\textrm{R}_{0}&=&\int_{0}^{\infty}\textrm{A}(\tau)d\tau
\end{eqnarray*}
Where
\begin{packed_enum}
\item j(t) is the number of new infections at time t
\item A($\tau$) is the rate an infected person infects a healthy person at time $\tau$ after infection
\item $\textrm{R}_{0}$ is the total number of new infections resulting from an infected person
\end{packed_enum}
If we assume the infection rate is constant for a number of consecutive days (w) following infection then
\begin{eqnarray*}
\textrm{j}(\textrm{t})&=&\textrm{A}(\textrm{t})\int_{0}^{\textrm{w}}\textrm{j}(\textrm{t}-\tau)d\tau \\
\textrm{R}_{0}(\textrm{t})&=&\textrm{w}\:\textrm{A}(\textrm{t})
\end{eqnarray*}
So
\begin{equation*}
\textrm{R}_{0}(\textrm{t})=\textrm{w}\:\frac{\textrm{j}(\textrm{t})}{\int_{0}^{\textrm{w}}\textrm{j}(\textrm{t}-\tau)d\tau}
\end{equation*}
Approximated as
\begin{eqnarray*}
\textrm{R}_{0}(\textrm{t})&=&\textrm{w}\:\frac{\textrm{j}(\textrm{t})}{\textrm{s}(\textrm{t})} \\
\textrm{s}(\textrm{t})&=&\sum_{\tau=0}^{\tau=\textrm{w}}\textrm{j}(\textrm{t}-\tau)
\end{eqnarray*}
\end{document}
